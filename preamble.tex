% TODO:
%  - Fix \deriv and \pderiv for inline math (mathchoice)
%     - Add mixed partial capability with \pderiv*
%     - Overall robustify
%     - OR find a good package that does this already
%  - Standardize this preamble
%     - csquotes
%     - biblatex
%     - siunitx
%     - microtype
%  - fancyhf -> plain (or new name, like "assignment")

\usepackage[LGR,T1]{fontenc}
\usepackage[greek,american]{babel}
\usepackage{blindtext}

\usepackage{mathtools,mleftright}
\usepackage{amssymb}
\usepackage{empheq}
\usepackage[version=4]{mhchem}

\usepackage[print-unity-mantissa=false]{siunitx}
\DeclareSIUnit\atmosphere{atm}
\DeclareSIUnit\sq{\ensuremath{\square}}

% Upright symbols
\makeatletter
\newcommand*{\eu}{\@ifnextchar^{\eu@}{\eu@^{}}}
\def\eu@^#1{%
  \expandafter\def\expandafter\eu@start\expandafter{\@car#1\@nil}%
  \ifnum\expandafter\ifx\eu@start\iu\relax 1
      \else\ifnum\expandafter\ifx\eu@start\ju\relax 1\else 0\fi\fi
    =1 %
    {\mathrm e}^{\,#1}%
  \else
    {\mathrm e}^{#1}%
  \fi
}
%
% \def\@eu^#1{%
%   \expandafter\def\expandafter\@eu@start\expandafter{\@car#1\@nil}%
%   \expandafter\ifx\@eu@start\iu\relax
%     \@@eu^{\,#1}%
%   \else
%     \expandafter\ifx\@eu@start\ju\relax
%       \@@eu^{\,#1}%
%     \else
%       \@@eu^{#1}%
%     \fi
%   \fi}
% \def\@@eu{\mathrm{e}}
\makeatother

\newcommand*{\iu}{\mathrm i}  % Imaginary unit i
\newcommand*{\ju}{\mathrm j}  % Imaginary unit j

\newcommand*{\ped}[1]{\ensuremath{_{\textnormal #1}}}  % Pedex
\newcommand*{\ap} [1]{\ensuremath{^{\textnormal #1}}}  % Apex

\newcommand*{\ssep}{\mid}  % Set separation symbol

% Operators
\DeclareMathOperator{\ent}{ent}            % ...
\DeclareMathOperator{\sgn}{sgn}            % Signum function

\DeclareMathOperator{\dom}  {dom}
\DeclareMathOperator{\codom}{codom}
\DeclareMathOperator{\imag} {Im}  % Overloaded

\DeclareMathOperator{\erf} {erf}            % Error function
\DeclareMathOperator{\erfc}{erfc}          % ...

\DeclareMathOperator{\bigOh}{\mathcal O}  % Big Oh

\renewcommand*{\Re}{\operatorname{Re}}     % Real part
\renewcommand*{\Im}{\operatorname{Im}}     % Imaginary part

\DeclareMathOperator{\csch}{csch}
\DeclareMathOperator{\sech}{sech}

\renewcommand*{\Pr}{\operatorname{Pr}}
\DeclareMathOperator{\expect}{E}
\DeclareMathOperator{\var}   {Var}
\DeclareMathOperator{\cov}   {Cov}

\makeatletter
\newcommand*{\diff}{\mathop{}\!\@diff}
\newcommand*{\deriv}[3][]{\mathchoice{\frac{\@diff^{#1}#2}{\@diff#3^{#1}}}
  {\diff^{#1}#2/\@diff#3^{#1}}
  {\diff^{#1}#2/\@diff#3^{#1}}
  {\diff^{#1}#2/\@diff#3^{#1}}}
\def\@diff{\mathrm{d}}
\newcommand*{\pdiff}{\mathop{}\!\@pdiff}
\newcommand*{\pderiv}[3][]{\mathchoice{\frac{\@pdiff^{#1}#2}{\@pdiff#3^{#1}}}
  {\pdiff^{#1}#2/\@pdiff#3^{#1}}
  {\pdiff^{#1}#2/\@pdiff#3^{#1}}
  {\pdiff^{#1}#2/\@pdiff#3^{#1}}}
\def\@pdiff{\partial}
\makeatother

\newcommand*{\dbar}{\mkern2mu\mathchar'26\mkern-11mu \mathrm{d}}
\newcommand*{\idiff}{\mathop{}\!\dbar}  % Inexact differential

\newcommand*{\Diff}{\mathop{}\!\mathrm{\Delta}}  % Increment

\DeclareMathOperator{\diag}{diag}
\newcommand*{\transpose}{^{\mathsf T}}

% Upright integrals from MathDesign
\makeatletter
\def\upintkern@{\mkern-7mu\mathchoice{\mkern-3.5mu}{}{}{}}
\def\upintdots@{\mathchoice{\mkern-4mu\@cdots\mkern-4mu}
  {{\cdotp}\mkern1.5mu{\cdotp}\mkern1.5mu{\cdotp}}
  {{\cdotp}\mkern1mu{\cdotp}\mkern1mu{\cdotp}}
  {{\cdotp}\mkern1mu{\cdotp}\mkern1mu{\cdotp}}}
\newcommand{\upiint}    {\DOTSI\protect\UpMultiIntegral{2}}
\newcommand{\upiiint}   {\DOTSI\protect\UpMultiIntegral{3}}
\newcommand{\upiiiint}  {\DOTSI\protect\UpMultiIntegral{4}}
\newcommand{\upidotsint}{\DOTSI\protect\UpMultiIntegral{0}}
\newcommand{\UpMultiIntegral}[1]{%
  \edef\ints@c{\noexpand\upintop
    \ifnum #1=\z@    \noexpand\upintdots@\else\noexpand\upintkern@\fi
    \ifnum #1>\tw@   \noexpand\upintop\noexpand\upintkern@\fi
    \ifnum #1>\thr@@ \noexpand\upintop\noexpand\upintkern@\fi
    \noexpand\upintop\noexpand\ilimits@}%
  \futurelet\@let@token\ints@a
}
\makeatother

\DeclareFontFamily{OMX}{mdbch}{}
\DeclareFontShape{OMX}{mdbch}{m} {n}{ <->s * [0.8]  mdbchr7v }{}
\DeclareFontShape{OMX}{mdbch}{b} {n}{ <->s * [0.8]  mdbchb7v }{}
\DeclareFontShape{OMX}{mdbch}{bx}{n}{<->ssub * mdbch/b/n}     {}

\DeclareSymbolFont{uplargesymbols}{OMX}{mdbch}{m}{n}
\SetSymbolFont{uplargesymbols}{bold}{OMX}{mdbch}{b}{n}
\DeclareMathSymbol{\upintop} {\mathop}{uplargesymbols} {82}
\DeclareMathSymbol{\upointop}{\mathop}{uplargesymbols}{"48}

\DeclareFontEncoding{MDB}{}{}
\DeclareFontFamily{MDB}{mdbch}{}
\DeclareFontShape{MDB}{mdbch}{m} {n}{ <->s * [0.8]  mdbchrmb }{}
\DeclareFontShape{MDB}{mdbch}{b} {n}{ <->s * [0.8]  mdbchbmb }{}
\DeclareFontShape{MDB}{mdbch}{bx}{n}{<->ssub * mdbch/b/n}     {}
\DeclareFontSubstitution{MDB}{cmr}{m}{n}
\DeclareSymbolFont{mathdesignB}{MDB}{mdbch}{m}{n}%
\SetSymbolFont{mathdesignB}{bold}{MDB}{mdbch}{b}{n}%
\DeclareMathSymbol{\upintclockwise}    {\mathop}{mathdesignB}{128}
\DeclareMathSymbol{\upointclockwise}   {\mathop}{mathdesignB}{130}
\DeclareMathSymbol{\upointctrclockwise}{\mathop}{mathdesignB}{132}
\DeclareMathSymbol{\upoiint}           {\mathop}{mathdesignB}{134}
\DeclareMathSymbol{\upoiiint}          {\mathop}{mathdesignB}{136}

\makeatletter
\newcommand{\upint} {\DOTSI\upintop\ilimits@}
\newcommand{\upoint}{\DOTSI\upointop\ilimits@}
\makeatother

% Eval right bar
\NewDocumentCommand{\evalat}{sO{\big}mm}{%
  \IfBooleanTF{#1}
    {\mleft. #3 \mright|_{#4}}
    {#3#2|_{#4}}%
}

% Nested display fractions with better vertical spacing
\NewDocumentCommand{\qfrac}{smm}{%
  \dfrac{\IfBooleanT{#1}{\vphantom{\big|}}#2}{\mathstrut #3}%
}

% Optional matrix vertical stretch and column alignment
\makeatletter
\renewcommand*{\env@matrix}[1][\arraystretch]{%
  \@ifnextchar[% ]
    {\env@matrix@i[{#1}]}
    {\env@matrix@i[{#1}][{*\c@MaxMatrixCols c}]}}
\def\env@matrix@i[#1][#2]{%
  \edef\arraystretch{#1}%
  \hskip -\arraycolsep
  \let\@ifnextchar\new@ifnextchar
  \array{#2}%
}
\makeatother

\usepackage{amsthm}
\newtheorem{problem} {Problem} [section]
\newtheorem{exercise}{Exercise}[section]
\makeatletter
\newenvironment{solution}[1][\solution@name]{\par
  \normalfont \topsep6\p@\@plus6\p@\relax
  \trivlist
  \item\relax
    {\itshape #1\@addpunct{.}}\hspace\labelsep\ignorespaces
}{\endtrivlist\@endpefalse}
\def\solution@name{Solution}
\makeatother

\usepackage{isomath}
\let\mathpi\pi
\renewcommand*{\pi}{\text{\textrm{\greektext p}}}

% Wide box for empheq
\newcommand*{\widefbox}[1]{\fbox{\hspace{1em}#1\hspace{1em}}}

% Lists
\usepackage[inline]{enumitem}

% Tables
\usepackage{booktabs}

% Figures and graphics
\usepackage{graphicx}
\usepackage{pgfplots}
\pgfplotsset{compat=newest}

% Header and footer
\usepackage{fancyhdr}
\usepackage{titling}

\makeatletter
\DeclareRobustCommand*{\course}[1]{\gdef\@course{#1}}
\DeclareRobustCommand*{\shorttitle}[1]{\gdef\@shorttitle{#1}}
\let\@course\@empty
\gdef\@shorttitle{\thetitle}
\newcommand{\makeheadfoot}{%
  \pagestyle{fancy}%
  \fancyhead[L,C]{}%
  \fancyhead[R]{\textbf\@course}%
  \fancyfoot[L]{\textsc\@shorttitle}%
  \fancyfoot[C]{\thepage}%
  \fancyfoot[R]{\textsc\theauthor}%
  \global\let\makeheadfoot\relax
  \global\let\@course\@empty
  \global\let\@shorttitle\@empty
  \global\let\course\relax
  \global\let\shorttitle\relax
}
\makeatother

% Symbolic footnote numbering
\renewcommand*{\thefootnote}{\fnsymbol{footnote}}

\usepackage[hidelinks]{hyperref}
