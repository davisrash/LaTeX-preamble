\let\mathpi\pi
\renewcommand*\pi{\textnormal{\greektext p}}
\newcommand*\iu{\mathrm{i}}
\newcommand*\eu{\mathrm{e}}

\DeclareMathOperator{\sgn}{sgn}

\renewcommand*\Re{\operatorname{Re}}
\renewcommand*\Im{\operatorname{Im}}

\DeclareMathOperator{\csch}{csch}
\DeclareMathOperator{\sech}{sech}

\DeclareMathOperator{\diag}{diag}
\DeclareMathOperator{\adj} {adj}
\newcommand*\transpose{^{\mathsf{T}}}

% Upright integrals from MathDesign
\makeatletter
\def\upintkern@{\mkern-7mu\mathchoice{\mkern-3.5mu}{}{}{}}
\def\upintdots@{\mathchoice{\mkern-4mu\@cdots\mkern-4mu}
  {{\cdotp}\mkern1.5mu{\cdotp}\mkern1.5mu{\cdotp}}
  {{\cdotp}\mkern1mu{\cdotp}\mkern1mu{\cdotp}}
  {{\cdotp}\mkern1mu{\cdotp}\mkern1mu{\cdotp}}}
\newcommand\upiint    {\DOTSI\protect\UpMultiIntegral{2}}
\newcommand\upiiint   {\DOTSI\protect\UpMultiIntegral{3}}
\newcommand\upiiiint  {\DOTSI\protect\UpMultiIntegral{4}}
\newcommand\upidotsint{\DOTSI\protect\UpMultiIntegral{0}}
\newcommand\UpMultiIntegral[1]{%
  \edef\ints@c{\noexpand\upintop
    \ifnum #1=\z@    \noexpand\upintdots@\else\noexpand\upintkern@\fi
    \ifnum #1>\tw@   \noexpand\upintop\noexpand\upintkern@\fi
    \ifnum #1>\thr@@ \noexpand\upintop\noexpand\upintkern@\fi
    \noexpand\upintop\noexpand\ilimits@}%
  \futurelet\@let@token\ints@a
}
\makeatother

\DeclareFontFamily{OMX}{mdbch}{}
\DeclareFontShape{OMX}{mdbch}{m} {n}{ <->s * [0.8]  mdbchr7v }{}
\DeclareFontShape{OMX}{mdbch}{b} {n}{ <->s * [0.8]  mdbchb7v }{}
\DeclareFontShape{OMX}{mdbch}{bx}{n}{<->ssub * mdbch/b/n}     {}

\DeclareSymbolFont{uplargesymbols}{OMX}{mdbch}{m}{n}
\SetSymbolFont{uplargesymbols}{bold}{OMX}{mdbch}{b}{n}
\DeclareMathSymbol{\upintop} {\mathop}{uplargesymbols} {82}
\DeclareMathSymbol{\upointop}{\mathop}{uplargesymbols}{"48}

\DeclareFontEncoding{MDB}{}{}
\DeclareFontFamily{MDB}{mdbch}{}
\DeclareFontShape{MDB}{mdbch}{m} {n}{ <->s * [0.8]  mdbchrmb }{}
\DeclareFontShape{MDB}{mdbch}{b} {n}{ <->s * [0.8]  mdbchbmb }{}
\DeclareFontShape{MDB}{mdbch}{bx}{n}{<->ssub * mdbch/b/n}     {}
\DeclareFontSubstitution{MDB}{cmr}{m}{n}
\DeclareSymbolFont{mathdesignB}{MDB}{mdbch}{m}{n}
\SetSymbolFont{mathdesignB}{bold}{MDB}{mdbch}{b}{n}
\DeclareMathSymbol{\upintclockwise}    {\mathop}{mathdesignB}{128}
\DeclareMathSymbol{\upointclockwise}   {\mathop}{mathdesignB}{130}
\DeclareMathSymbol{\upointctrclockwise}{\mathop}{mathdesignB}{132}
\DeclareMathSymbol{\upoiint}           {\mathop}{mathdesignB}{134}
\DeclareMathSymbol{\upoiiint}          {\mathop}{mathdesignB}{136}

\makeatletter
\newcommand\upint {\DOTSI\upintop\ilimits@}
\newcommand\upoint{\DOTSI\upointop\ilimits@}
\makeatother

\renewcommand*\int{\upint}

% Nested display fractions with better vertical spacing
\NewDocumentCommand{\qfrac}{smm}{%
  \dfrac{\IfBooleanT{#1}{\vphantom{\big|}}#2}{\mathstrut #3}%
}

% Optional matrix vertical stretch and column alignment
\makeatletter
\renewcommand\env@matrix[1][\arraystretch]{%
  \@ifnextchar[% ]
    {\env@matrix@i[{#1}]}
    {\env@matrix@i[{#1}][{*\c@MaxMatrixCols c}]}}
\def\env@matrix@i[#1][#2]{%
  \edef\arraystretch{#1}%
  \hskip -\arraycolsep
  \let\@ifnextchar\new@ifnextchar
  \array{#2}}
\makeatother

\newtheorem{problem}{Problem}[section]
\newtheorem{exercise}{Exercise}
\makeatletter
\newenvironment{solution}[1][\solution@name]{\par
  \normalfont \topsep6\p@\@plus6\p@\relax
  \trivlist
  \item\relax
    {\itshape #1\@addpunct{.}}\hspace\labelsep\ignorespaces
}{\endtrivlist\@endpefalse}
\def\solution@name{Solution}
\makeatother

% Wide box for empheq
\newcommand*\widefbox[1]{\fbox{\hspace{1em}#1\hspace{1em}}}

\makeatletter
\DeclareRobustCommand*\course[1]{\gdef\@course{#1}}
\DeclareRobustCommand*\shorttitle[1]{\gdef\@shorttitle{#1}}
\let\@course\@empty
\gdef\@shorttitle{\thetitle}
\pagestyle{fancy}
\fancyhead[L,C]{}
\fancyhead[R]{\textbf\@course}
\fancyfoot[L]{\textsc\@shorttitle}
\fancyfoot[C]{\thepage}
\fancyfoot[R]{\textsc\theauthor}
\makeatother

% Symbolic footnote numbering
\renewcommand*\thefootnote{\fnsymbol{footnote}}
